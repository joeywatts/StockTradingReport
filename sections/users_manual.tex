\section{User's Manual}

In this section, we describe the usage of our trading simulation software, including setting up databases, using the opinion aggregation implementation, and executing backtests for trading strategies.

\subsection{Usage Environment}

Our software does not come with a graphical user interface or even a single command line entrypoint.
Our code is written with Scala \cite{scala} and uses sbt \cite{sbt} to manage project structure and dependencies.
Additionally, our system depends upon an HBase \cite{hbase} instance or cluster.
We have been provided a Hadoop cluster that runs HBase by Virginia Tech, so this manual will assume that you have an HBase instance set up and that you have sbt installed.
Please consult an external reference such as \textit{Deploying HBase on a Cluster} by Cloudera \cite{clouderaHbaseDeploy} if you need any assistance setting this up.
If you are affiliated with Virginia Tech and are interested in the DLRL Hadoop cluster that we used, please contact Prof. Edward Fox for more information.

\subsection{Use Cases}

\subsubsection{Populating Databases}

Before you can run any simulations, you must populate HBase with stock prices for the date range of interest.
Additionally, if you want to use our sentiment-based strategies, you must populate HBase with microblog posts.
To populate HBase with stock prices, you can use our \texttt{YahooGoogleFinance\-DBPopulate} utility, which will query Yahoo Finance and Google Finance for daily stock price summaries on a given date range and populate the database with the results.
There are two variables, \texttt{startDate} and \texttt{endDate}, which define the date range of daily stock prices to query.
You should avoid setting a range that spans more than one year due to the limitations of the respective APIs.
You can configure the symbols to query from Yahoo Finance and Google Finance by modifying the two lists named \texttt{yahooSymbols} and \texttt{googleSymbols}.
The \texttt{dataSource} variable specifies that the \texttt{YahooFinance} table should be used.
Therefore, you should create the HBase table named \texttt{stockprices\_yahoo} with the \texttt{price} column family (by executing \texttt{create `stockprices\_yahoo', `price'} in the HBase shell).
For more information, see the Design report in Appendix~\ref{design}.
To run the utility, use the command \texttt{sbt ``PricingData/run-main\\ cs4624.prices.test.YahooGoogleFinanceDBPopulate''}.

\subsubsection{Stock Opinion Aggregation}

We implemented the opinion aggregation model from a paper called \textit{CrowdIQ} \cite{crowdiq}.
A library to use our implementation is provided with the Opinion Aggregation subproject.
The \texttt{AggregatedOpinions} class provides an interface that will compute the aggregated sentiment towards a certain stock.
This class requires that you have a source of stock price data.
Additionally, you must specify a window of time which determines the amount of time after a post that we will confirm the accuracy of the post and adjust the weight for the post's author.
To use an instance of this class, you pass the microblog posts in the interval (i.e., between market open events) to the \texttt{on} method.
Using the \texttt{sentimentForStock} method, you can get the aggregated sentiment (bullish, bearish, or inconclusive) towards a stock symbol.
After you've gotten the sentiments for that date, you use the \texttt{reset} method to start a new interval of posts and repeat the process.

\subsubsection{Backtesting Trading Strategies}

Our software provides the ability to test trading strategies by replaying historical data (a ``backtest'').
Trading strategies are defined as extensions of the \texttt{Trading\-Strategy} trait.
All data that a trading strategy can use to make decisions is formed into a class that extends the \texttt{Trading\-Event} trait.
In a backtest, these events are collected for a particular time interval and passed in-order to each trading strategy by an instance of the \texttt{Trading\-Context} class.
The \texttt{Trading\-Context} class takes the sources of these trading events, the set of trading strategies, and the time interval in its constructor.
The \texttt{run} method, which will execute the backtest, takes a callback which allows you to see the change in each strategy's portfolio after an event.
Our tests are written in the \texttt{Trading\-Simulation.scala} file.
To execute this main class, run \texttt{sbt ``TradingSimulation/run-main\allowbreak cs4624.trading.TradingSimulation''}.

%%% Local Variables:
%%% mode: latex
%%% TeX-master: "../report"
%%% End:
