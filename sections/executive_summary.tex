
\section{Executive Summary}

The goal of this project is to leverage microblogging data about the stock market to predict price trends and execute trades based on these predictions.
Predicting the price trends of stocks with microblogging data involves a complex opinion aggregation model.
For this, we built upon previous research, specifically a paper called ``CrowdIQ'' submitted by a team consisting of some Virginia Tech faculty \cite{crowdiq}.
This paper details a complicated method of aggregating an accurate opinion by modeling judge reliability and interdependence.
Once the overall sentiment of the judges was deduced, we built trading strategies that take this information into account to execute trades.

The first step of the project was a sentiment analysis of posts on a microblogging site named StockTwits \cite{stocktwits}.
These messages can contain a label indicating a bullish or bearish sentiment, which will help indicate a specific position to take on a given stock.
However, most users choose not to use these labels on their StockTwits \cite{crowdiq}.
A classification of these unlabeled tweets is required to autonomously utilize StockTwits to drive the proposed trading strategies.

With a working sentiment analysis model, we implemented the opinion aggregation model described by \textit{CrowdIQ} \cite{crowdiq}.
This can gauge an accurate market sentiment for a particular stock based on the collection of sentiments that are received from users on StockTwits.

The next step was the creation of a trading simulation platform, including a complete virtual portfolio management system and an API for retrieving historical and current stock data.
These tools allow us to run quick and repeatable tests of our trading strategies on historical data.
We can easily compare the performance of strategies by running them with the same historical data.

After we had a viable testing environment setup, we implemented trading strategies.
This required research and analysis of other attempts at similar uses of microblogging data on predicting stock returns.
The testing environment was focused on a set of stocks that is consistent with those used in \textit{CrowdIQ}.
The implementation of the \textit{CrowdIQ} strategy served as a baseline against which we compared our results.

Development of new trading strategies is an open-ended task that involved a process of trial and error.
It is possible for a strategy to find success in 2014, but not perform quite as well in other years, because market climates can be fickle.
To assess the dependence of the market climate on our strategy's success, we also tested against data for the year of 2015 and compared the performance.

The final deliverable is a viable trading simulation environment coupled with various trading strategies and an analysis of their performance in the years of 2014 and 2015.
The analysis of each strategy's performance indicated that our sentiment-based strategies perform better than the index in bullish markets like that of 2014, but, when they encounter a bear market, they typically make poor trading decisions which result in a loss of value.

%%% Local Variables:
%%% mode: latex
%%% TeX-master: "../report"
%%% End:
