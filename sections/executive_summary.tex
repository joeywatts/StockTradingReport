
\section{Executive Summary}

The goal of this project is to leverage microblogging data about the stock market to predict price trends and execute trades based on these predictions. Predicting the price trends of stocks with microblogging data involves a complex opinion aggregation model. For this, we are building upon previous research, specifically a paper called ``CrowdIQ'' submitted by a team consisting of some Virginia Tech faculty \cite{crowdiq}. This paper details a complicated method of aggregating an accurate opinion by modeling judge reliability and interdependence. Once the overall sentiment of the judge's is deduced, we can build trading strategies that take this information into account to execute trades.

The first step of the project is a sentiment analysis of these StockTwits. These messages can contain a label indicating a bullish or bearish sentiment, which will help indicate a specific position to take on a given stock. However, most users choose not to use these labels on their StockTwits. A classification of these unlabeled tweets is the first step towards utilizing these StockTwits at an automatic level to drive the proposed trading strategies.

The next step is creating a virtual portfolio management system and an API for retrieving historical and current stock data. These two tools will allow us to run quick and repeatable tests of our trading strategies on historical data instead of having to wait for real time data. This also acts as a filter for any trading strategies that perform below our baseline.

After we have a viable testing environment setup we can began to implement trading strategies. This will require research into other attempts at similar uses of microblogging data on predicting stock returns. Doing so will give us ideas on trading strategies that might be predictable based on StockTwits sentiment, i.e. short selling. The testing environment will be focused on a set of stocks from 2014, as this is the same data set the CrowdIQ paper mentioned earlier used. This will give us a baseline to compare our methods to.

As the stock market can be somewhat volatile we hope to develop a wide array of trading strategies. We realize that the data set we are using initially will not give a completely accurate representation of a similar time frame at any other point in time. However, we hope to use it to identify a selection of trading strategies that we will implement in real time. This is the last step in the project and will take a majority of the time as it will be an iterative process. There will always be room for improvement on any trading strategy we devise.

The final deliverable will be a concrete trading strategy that acts based on real-time sentiment of stocks as derived from micro-blogging data from StockTwits. This model will be accompanied by statistically significant experiment results and explanation on our methods' advantages over other papers. This will most likely be shown in the form of comparisons to other methods developed internally and externally, such as in other papers.

%%% Local Variables:
%%% mode: latex
%%% TeX-master: "../report"
%%% End:
